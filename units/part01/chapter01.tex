\providecommand{\main}{../..}
\documentclass[\main/notes.tex]{subfiles}

\begin{document}
	\chapter[Introduction]{Introduction to Business Information Systems}
		\section{What is an Information System?}
			\begin{definition}{System}
				A set of elements or components that interact together to accomplish goals.
			\end{definition}
			A system consists of four components:
			\begin{description}
				\item[Input] The data to be processed --- the activity of gathering and capturing data
				\item[Processing] How the data must be processed --- converting or transforming input into useful outputs
				\item[Output] The result of processing the data --- production of useful information, often in the form of documents or reports
				\item[Feedback] Assessment of the output --- Output that is used to make changes to input or processing activities
			\end{description}
			\subsection{System Performance and Feedback}
				The last stage of a system is feedback. This can be measured in different ways.
				\begin{definition}{Efficiency}
					A measure of what is produced, divided by what is consumed.
				\end{definition}
				\begin{definition}{Effectiveness}
					A measure of the extent to which a system achieves its goals. It can be computed by dividing the goals actually achieved by the total of the stated goals.
				\end{definition}
				\begin{definition}{System Performance Standard}
					A specific objective of the system.
				\end{definition}
			\subsection{What Is Information}
				\begin{definition}{Information}
					A collection of facts
				\end{definition}
				Data goes into a system, and information comes out.
			\subsection{What is an Information System?}
				\begin{definition}{Information System}
					A set of interrelated components that collect, manipulate, store and disseminate information, and provide a feedback mechanism, to meet an objective.
				\end{definition}
				\begin{definition}{Forecasting}
					Predicting future events
				\end{definition}
			\subsection{Characteristics of Valuable Information}
				The value of information is directly related to how it helps decision makers achieve their organisation's goals.
				\begin{center}
					\begin{tblr}{colspec={lX}, row{1}={c, font=\bfseries}}
						\toprule
						Characteristic & Definition\\
						\midrule
						Accessible & Authorised users can easily access information in the right format at the right time\\
						Accurate & Information is error free\\
						Complete & Contains all the important facts, but not extra information\\
						Economical & Information is not costly to produce --- balance the value of the information with the cost to produce it\\
						Flexible & Information can be used for a variety of purposes\\
						Relevant & Doesn't contain information that is useless or unimportant to the decision maker\\
						Reliable & Information can be depended on\\
						Secure & Information cannot be accessed by unauthorised users\\
						Simple & Information is not overly complex\\
						Timely & Information is delivered when needed\\
						Verifiable & Information can be checked to make sure it is correct\\
						\bottomrule
					\end{tblr}
				\end{center}
			\pagebreak
			\subsection{Manual and Computerised Information Systems}
				\begin{definition}{Computer-Based Information System (CBIS)}
					A single set of \concept{hardware}, \concept{software}, \concept{databases}, \concept{telecommunications}, \concept{people} and \concept{procedures} that is configured to collect, manipulate, store and process data into information. 
				\end{definition}
				\subsubsection*{The Components of a Computer-Based Information System}
					\begin{description}
						\item[Hardware] Any machinery that assists in the input, processing, storage and output activities of an information system. 
						\item[Software] The computer programs that govern the operation of the computer. There are two types: \concept{system software}, which controls basic computer operations, and \concept{applications software}, which allows a user to accomplish specific tasks.
						\item[Databases] An organised collection of electronic information.
						\item[Telecommunications] The electronic transmission of signals for communications, which enables organisations to carry out processes and tasks through effective computer networks. This involves:
							\begin{description}
								\item[Networks] Computers and equipment that are connected in a building, around the country, or around the world, to enable electronic communication.
								\item[Internet] The world's largest computer network, consists of thousands of interconnected networks that freely exchange information.
								\item[World Wide Web (WWW)] A network of links on the internet to documents containing text, graphics, video and sound. Information about the documents and access to them is controlled by special computers called \concept{web servers}.
								\item[Cloud Computing] A computing environment where software and storage are provided as an Internet service, and are accessed via a web browser.
								\item[Intranets] An internal company network, built using Internet and World Wide Web standards and products, that allows people within an organisation to exchange information and work on projects.
								\item[Extranet] A network based on web technologies that allows selected outsiders to access authorised resources of a company's intranet.
							\end{description}
						\item[People] The most important element in computer-based information systems. Involves users of the system, and information systems personnel.
						\item[Procedures] The strategies, policies, methods and rules for using a Computer-Based Information System (CBIS). 
					\end{description}
		\pagebreak
		\section{Business Information Systems}
			\subsection{Systems at Different Levels}
					\begin{description}
						\item[Operational Level] Support the day-to-day running of the firm
							\begin{itemize}
								\item Transaction Processing Systems
								\item Customer relationship management systems
								\item Supply chain management systems
							\end{itemize}
						\item[Technical Level] Support midterm decisions made by middle managers
							\begin{itemize}
								\item Management Information Systems
								\item Decision Support Systems
							\end{itemize}
						\item[Strategic Level] Support long-term, strategic decisions made by senior managers
							\begin{itemize}
								\item Executive Support Systems
							\end{itemize}
					\end{description}
			\subsection{Different System Types}
				\subsubsection{Transaction Processing Systems}
					\begin{definition}{Transaction}
						Any business-related exchange, such as payments to employees, sales to customers and payments to suppliers.
					\end{definition}
					\begin{definition}{Transaction Processing System (TPS)}
						An organised collection of people, procedures, software, databases and devices used to record completed business transactions.
					\end{definition}
					\begin{definition}{E-commerce}
						Any business transaction executed electronically, either between
						\begin{itemize}
							\item companies (business-to-business, B2B)
							\item companies and consumers (business-to-consumer, B2C)
							\item consumers and other consumers (consumer-to-consumer, C2C)
							\item companies and the public sector, or
							\item consumers and the public sector
						\end{itemize}
					\end{definition}
					\begin{definition}{Mobile Commerce (m-commerce)}
						Conducting business transactions electronically using mobile devices such as smartphones
					\end{definition}
					\begin{definition}{Electronic Business (e-business)}
						Using information systems and the Internet to perform all business-related tasks and functions.
					\end{definition}
				\subsubsection{Enterprise Resource Planning}
					\begin{definition}{Enterprise Resource Planning (ERP) System}
						A set of integrated programs, that manages the vital business operations for an entire multisite, global organisation. 
					\end{definition}
				\subsubsection{Management Information Systems}
					\begin{definition}{Management Information System (MIS)}
						An organised collection of people, procedures, software, databases and devices that provides routine information to managers and decision makers.\\
						An MIS focuses on operational efficiency. It typically provides standard reports generated with data and information from the TPS. The output of a TPS is the input to an MIS.
					\end{definition}
				\subsubsection{Decision Support Systems}
					\begin{definition}{Decision Support System (DSS)}
						An organised collection of people, procedures, software, databases and devices that support problem-specific decision-making.\\
						A DSS focuses on making effective decisions.
					\end{definition}
					The essential elements of a DSS include:
					\begin{indentparagraph}
						\begin{description}
							\item[model base] a collection of models used to support a decision maker. Often managed using a \concept{model management system (MMS)}.
							\item[database] a collection of facts and information to assist in decision-making. Often managed using a \concept{database management system (DBMS)}.
							\item[dialogue manager or user interface] systems and procedures that help decision makers interact with the DSS.
						\end{description}
					\end{indentparagraph}
					\begin{definition}{Group Decision Support System}
						Also called a \concept{group support system}. Includes the DSS elements of above, as well as software called \concept{groupware} used to help groups make effective decisions.
					\end{definition}
					\begin{definition}{Executive Support System}
						Also called an \concept{executive information system}. Helps top-level managers make better decisions.
					\end{definition}
				\subsubsection{Knowledge Management Systems}
					\begin{definition}{Knowledge Management System}
						An organised collection of people, procedures, software, databases and devices used to create, store, share and use the organisation's knowledge and experience.
					\end{definition}
					\begin{definition}{Artifical Intelligence}
						The ability of computer systems to mimic or duplicate the functions or characteristics of the human brain or intelligence.
					\end{definition}
					The major elements of \concept{Artificial Intelligence} are:
					\begin{indentparagraph}
						\begin{description}
							\item[Robotics] an area of AI in which machines take over complex, dangerous or boring tasks
							\item[Vision Systems] allow robots and other devices to `see', store and process visual images
							\item[Natural Language Processing] computers ability to understand and possibly act on verbal or written commands in human languages
							\item[Learning Systems] allow computers to learn from past mistakes or experiences
							\item[Neural Networks] allow computers to recognise and act on patterns or trends
							\item[Expert Systems] give computers the ability to make suggestions and act like an expert in a particular field 
						\end{description}
					\end{indentparagraph}
					\begin{definition}{Expert Systems}
						Hardware and software that stores knowledge and makes inferences, similar to a human expert.
					\end{definition}
					\begin{definition}{Knowledge Base}
						A component of an expert system that stores all relevant information, data, rules, cases and relationships used by the expert system
					\end{definition}
					\begin{definition}{Virtual Reality}
						The simulation of a real or imagined environment that can be experienced visually in three dimensions.
					\end{definition}
			\pagebreak
			\section{Systems Development}
				\begin{definition}{Systems Development}
					The activity of creating or modifying existing business systems.
				\end{definition}
				\begin{enumerate}
					\itemname[Systems Investigation:] Understand the problem
					\itemname[Systems Analysis:] Determine what must be done to solve the problem
					\itemname[Systems Design:] The solution is planned out
					\itemname[Systems Implementation:] The solution is built or bought, and replaces the old system
					\itemname[Systems Maintenance and Review:] The new system is evaluated 
				\end{enumerate}
			\section{Computer and Information Systems Literacy}
				\begin{definition}{Computer Literacy}
					Knowledge of computer systems and equipment and the ways they function. It stresses equipment and devices (hardware), programs and instructions (software), databases and telecommunications.
				\end{definition}
				\begin{definition}{Information Systems Literacy}
					Knowledge of how data and information are used by individuals, groups, and organisations.
				\end{definition}
			\section{Global Challenges in Information Systems}
				\begin{description}
					\item[Cultural challenges] Different places have different cultures and customs which can significantly affect individuals and organisations. 
					\item[Language challenges] Differences in languages make exact translations challenging.
					\item[Time and distance challenges] Large time differences make it difficult to talk to people in different locations. Long distances make it difficult to get required equipment timeously.
					\item[Infrastructure challenges] Certain parts of the world may not have easy access to required utilities.
					\item[Currency challenges] Exchange rates change, which complicates international trade.
					\item[Product and service challenges] Physical products can be difficult to deliver globally.
					\item[Technology transfer issues] Certain equipment and systems (for example, military-grade) cannot be sold to some countries.
					\item[National laws] Keeping track of different laws in different countries can be difficult.
					\item[Trade agreements] 
				\end{description}
				\pagebreak
			\section{Exercises}
				\begin{exercise}{Self-Assessment}
					\begin{enumerate}
						\item Input is \concept{processed} to produce output.
						\item A \concept{system performance standard} is a specific objective of a system.
						\item \concept{Feedback} is information from a system that is used to make changes to the input.
						\item Providing software and data storage via a web browser is usually called \concept{cloud computing}.
						\item The most important element in an information system is the \concept{people}.
						\item An \concept{executive information system} supports long-term strategic decision-making.
						\item A business related exchange is known as a \concept{transaction}.
						\item M-commerce involves paying for goods and services using a \concept{mobile device}.
						\item The activity of creating an information system is called \concept{systems development}.
						\item When a person has the ability to use a computer, they are known as being \concept{computer-literate}.
					\end{enumerate}
				\end{exercise}
				\begin{exercise}{Review Questions}
					\begin{enumerate}
						\question{Describe what an information system is. What are its main elements?} A set of elements that interact together to produce and process information, to accomplish a goal. The elements of an IS are: input (data), processing, output (information), and feedback.
						\question{Define the word `system'} A set of elements or components that interact together to achieve a goal.
						\question{What are the main components of a system? What does each of them do?} Input, Processing, Output, Feedback. Input is the data to be processed, processing is transforming the data into useful information, output is the result of the processing, and feedback is analysis of the operation of the system.
						\question{Describe what a TPS does.} A Transaction Processing System records completed business transactions.
						\question{What is m-commerce?} Conducting business transactions electronically using mobile devices.
						\question{Define telecommunications.} The electronic transmission of signals for communications.
						\question{What are the main steps in systems development?} Systems Investigation, Systems Analysis, Systems Design, Systems Implementation, Systems Maintenance and Review.
					\end{enumerate}
				\end{exercise}
		\vbox{\rulechapterend}
\end{document}
