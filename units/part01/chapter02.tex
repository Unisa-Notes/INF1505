\providecommand{\main}{../..}
\documentclass[\main/notes.tex]{subfiles}

\begin{document}
	\setcounter{chapter}{1}
	\chapter{Business Information Systems in Organisations}
		\chaptermark{Organisations Business Info Systems}
		\section{Introduction to Organisations}
			\begin{definition}{Organisation}
				A formal collection of people and other resources established to accomplish a set of goals.

				A system, which therefore has inputs, processing mechanisms, outputs, and feedback.
			\end{definition}
			\subsection{Value Chain}
				\begin{definition}{Value Chain}
					A series (chain) of activities that includes inbound logistics, warehouse and storage, production, finished product storage, outbound logistics, marketing and sales, and customer service.
				\end{definition}
				\begin{itemize}
					\item Analysing these results in efficient TPS, an expanding market, and the sharing of information. Used to examine what happens to raw materials to add value to them, before the finished product is sold.
					\item Can also reveal linkages between different activities, which can then be exploited using an information system.
				\end{itemize}
				\begin{definition}{Supply Chain Management (SCM)}
					Key part of managing the value chain. Helps determine what supplies are needed for the value chain, what quantities are needed to meet customer demand, how the supplies should be processed into finished goods and services, and how the shipment of supplies and products to customers should be schedules, monitored, and controlled.
				\end{definition}
				\begin{definition}{Customer Relationship Management (CRM)}
					Key part of managing the value chain. Helps a company manage all aspects of customer encounters.
				\end{definition}
			\subsection{Organisational Structure}
				\begin{definition}{Organisational Structure}
					The organisational subunits, and the way they relate to each other.

					Depends on its approach to management, and can affect how it views and uses information systems.

					Types: traditional, project, team, and virtual.
				\end{definition}
				\subsubsection{Traditional Organisational Structure}
					\begin{definition}{Traditional Organisational Structure}
						Also called a \concept{hierarchical structure}. Similar to a managerial pyramid, where the hierarchy of decision-making and authority flows from the strategic management at the top, down to operational management, and non-management employees. The strategic level has a higher degree of decision authority, more impact on business goals, and more unique problems to solve.
						\begin{center}
							\begin{forest}
								forked edges,
								for tree={draw,align=center,edge={-latex}, rounded corners}
								[CEO or Executive
									[Sales and \\ Marketing Manager
										[Marketing]
										[Sales]
									]
									[Research and \\ Development]
									[Operations \\ Manager
										[Human Resources]
										[Administration]
									]
								]
							\end{forest}
						\end{center}
						\begin{description}
							\item[Line Position] Positions or departments directly associated with making, packing, or shipping goods.
							\item[Staff Position] Positions not directly involved with the formal chain of command, but instead assist a department or area.
						\end{description}
					\end{definition}
					\begin{definition}{Flat Organisational Structure}
						An organisational structure with a reduced number of management layers.

						Empowers employees at lower levels to make decisions and solve problems without needing permission from mid-level managers.
					\end{definition}
					\begin{definition}{Empowerment}
						Giving employees and their managers more responsibility and the authority to make decisions, take certain actions, and have more control over their jobs.
					\end{definition}
				\subsubsection{Project and Team Organisational Structure}
					\begin{definition}{Project Organisational Structure}
						A structure centred on major products or services.
						\begin{center}
							\begin{forest}
								forked edges,
								for tree={grow=0, reversed, draw,align=center,edge={-latex}, parent anchor=east,child anchor=west, rounded corners}
								[Managing Director
									[Board Director \\ Project 1
										[Finance Director]
										[Marketing  Director]
										[Production  Director]
										[Sales  Director]
									]
									[Board Director \\ Project 2
										[Finance Director]
										[Marketing  Director]
										[Production  Director]
										[Sales  Director]
									]
								]
							\end{forest}
						\end{center}
					\end{definition}
					\begin{definition}{Team Organisational Structure}
						A structure centred on work teams or groups.
					\end{definition}
				\subsubsection{Virtual Organisational Structure}
					\begin{definition}{Virtual Organisational Structure}
						A structure that employs individuals, groups, or complete business units, in geographically dispersed areas, that can last for a few weeks or years, often requiring telecommunications on the Internet.

						\begin{description}
							\item[Advantages] Reduce costs, increase revenue, extra level of security.
						\end{description}
					\end{definition}
			\pagebreak
			\subsection{Organisational Change}
				\begin{definition}{Organisational Change}
					The responses that are necessary so that for-profit and non-profit organisations can plan for, implement, and handle changes.
					\begin{description}[nosep]
						\item[Sustaining Change] Changes that help an organisation improve its operations
						\item[Disruptive Change] Changes that harm an organisation's performance, and can even put it out of business.
					\end{description}
				\end{definition}
				\begin{sidenote}{Strategies to contain costs}
					In order to contain the costs associated with hiring, training, and compensating talented staff, the following strategies are used:
					\begin{description}[nosep]
						\item[Outsourcing] Contracting with outside professional services to meet specific business needs.
						\item[On-demand computing] Extension of the outsourcing approach. Contracting for computer resources to rapidly respond to an organisation's varying workflow. Also called \concept{on-demand business} and \concept{utility computing}.
						\item[Downsizing] Reduce the number of employees to cut costs. Also called \concept{rightsizing}.
					\end{description}
				\end{sidenote}
				\begin{definition}{Organisational Learning}
					The adaptations to new conditions or alterations of organisational practices over time.
				\end{definition}
			\subsection{Reengineering and Continuous Improvement}
				\begin{definition}{Reengineering}
					Also called \concept{process redesign}, and \concept{business process reengineering (BPR)}. The radical redesign of business processes, organisational structures, information systems, and values of the organisation, to achieve a breakthrough in business results.
				\end{definition}
				\begin{definition}{Continuous Improvement}
					Constantly seeking ways to improve business processes to add value to products and services.
				\end{definition}
				\begin{center}
					\begin{tblr}{colspec={>{\raggedright}X>{\raggedright}X}, row{1}={font=\bfseries}}
						\toprule
						Business Process Reengineering & Continuous Improvement\\
						\midrule
						Strong action taken to solve serious problems & Routine action taken to make minor improvements\\
						Top-down change driven by senior executives & Bottom-up change driven by workers\\
						Broad in scope, cuts across departments & Narrow in scope, focus is on tasks in a given area\\
						Goal is to achieve a major breakthrough & Goal is continuous, gradual improvements\\
						Often led by outsiders & Often led by workers close to the business\\
						Information system integral to the solution & Information system provides data to guide the improvement team\\
						\bottomrule
					\end{tblr}
				\end{center}

			\subsection{User Satisfaction and Technology Acceptance}
				Use \concept{technology diffusion} and \concept{technology infusion} to determine the actual usage of an information system.
				\begin{definition}{Technology Diffusion}
					A measure of how widely technology is spread throughout an organisation.

					An organisation has a high level of technology diffusion if computer and information systems are located in most departments.
				\end{definition}
				\begin{definition}{Technology Infusion}
					The extent to which technology is deeply integrated into an area or department. 
				\end{definition}
				\begin{sidenote}{Technology, Organisation and Environment (TOE) Framework}
					Diffusion and infusion depend on the technology available now and in the future, the size and type of the organisation, and the environmental factors (including competition, government regulations and so on).
				\end{sidenote}

			\subsection{The Applications Portfolio}
				\begin{definition}{Applications Portfolio}
					A scheme for classifying systems according to the contribution they make to the organisation. There are four types:
					\begin{description}
						\item[Support Applications] Make work more convenient, but are not essential.
						\item[Key Operational Applications] Applications essential to the organisation -- without them, it would not be able to do business
						\item[Strategic Application] An information system that gives a business an advantage over some or all of its competitors.
						\item[Future Strategic Application] Also called \concept{potential strategic}, or \concept{high potential}. Ideas for systems, which, if fully developed and deployed, might one day become strategic applications.
					\end{description}
				\end{definition}
			\pagebreak
			\subsection{Success Factors}
				In order for an IS to be implemented successfully, it needs to be aligned with the company's goals.
				\begin{definition}{Alignment}
					When the output from an information system is exactly what is needed to help a company achieve its strategic goals.

					To achieve alignment, consider the business processes already in place, and what information systems can be used to support these systems.
				\end{definition}
				\begin{definition}{Requirements Engineering}
					Also known as \concept{requirements analysis} and \concept{requirements capture}. Identifying what an information system is needed to do.
				\end{definition}

		\section{Competitive Advantage}
			\begin{definition}{Competitive Advantage}
				The ability of a firm to outperform its industry -- that is, to earn a higher rate of profit than the industry norm.
			\end{definition}
			\subsection[Competitive Advantage Causes]{Factors That Lead Firms to Seek Competitive Advantage}
				\begin{definition}{Five-Forces Model}
					A widely accepted model that identifies five key factors that can lead to attainment of competitive advantage. The more of the forces that are combined in any instance, the more likely firms will seek competitive advantage.
				\end{definition}
				\begin{sidenote}{Five-Forces Model Factors}
					\begin{enumerate}
						\item Rivalry among existing competitors
						\item Threat of new entrants
						\item Threat of substitute products and services
						\item Bargaining power of buyers
						\item Bargaining power of suppliers
					\end{enumerate}
				\end{sidenote}
				\subsubsection{Strategies for Competitive Advantage}
					\begin{sidenote}{Strategies}
						\begin{description}
							\item[Cost Leadership] Deliver the lowest possible products and services costs.
							\item[Differentiation] Deliver different products and services.
							\item[Niche Strategy] Deliver to only a small, niche market.
							\item[Altering the Industry Structure] Change the industry to become more favourable to the company or organisation. This can possibly be done using \concept{strategic alliances}.
								\begin{description}
									\item[Strategic Alliance] Also known as a \concept{strategic partnership}. An agreement between two or more companies that involves the joint production and distribution of goods and services.
								\end{description}
							\item[Creating New Products and Services] Introduce new products and services periodically or frequently.
							\item[Improving Existing Product Lines and Services] Make real of perceived improvements to existing product lines and services.
						\end{description}
					\end{sidenote}

		\section{Evaluating IS}
			There are different measures that can be used to evaluate the contributions information systems make to an organisation.
			\subsection{Productivity}
				\begin{definition}{Productivity}
					A measure of the output achieved divided by the input required.
				\end{definition}
			\subsection{Return on Investment and the Value of Information Systems}
				\begin{description}
					\item[Return on Investment (ROI)] Measure the additional profits or benefits that are generated as a percentage of the investment in IS technology. Can be difficult to measure.
					\item[Earnings Growth] Measure the increase in profit or earnings growth.
					\item[Market Share] The percentage of sales that a product or service has in relation to the total market.
					\item[Customer Awareness and Satisfaction] Measure the performance of an IS based on feedback from internal and external users.
					\item[Total Cost of Ownership (TCO)] Measure the total cost of owning computer equipment, including desktop computers, networks, and large computers.
				\end{description}
				\begin{sidenote}{Risk}
					The costs of development and implementation can be greater than the returns of a new system.
				\end{sidenote}

		\section{Careers in Information Systems}
			The IS department has three primary responsibilities: \concept{operations}, \concept{systems development}, and \concept{support}.
			\begin{sidenote}{Operations}
				Focus more on the efficiency of IS functions rather than their effectiveness. Primarily run and maintain IS equipment.
				\begin{description}
					\item[Career] Systems operators
				\end{description}
			\end{sidenote}
			\begin{sidenote}{Systems Development}
				Focus on the development of specific projects, and ongoing maintenance and review.
				\begin{description}
					\item[Career] Systems analysts and programmers
				\end{description}
				An analyst helps determine what outputs are needed from the system, and constructs plans to develop the necessary programs. This plan is then worked on with a programmer. Often use an \concept{agile approach to software development}, where software is written rapidly then continuously improved. One such approach is called \concept{Scrum}.
			\end{sidenote}
			\begin{sidenote}{Support}
				Provides user assistance in hardware and software acquisition and use, data administration, user training and assistance, and web administration.
			\end{sidenote}
			\begin{definition}{Information Service Units}
				A miniature IS department attached and directly reporting to a functional area in a large organisation.
			\end{definition}
			\subsection{Typical IS Titles and Functions}
				\begin{description}
					\item[Chief Information Officer (CIO)] Employ an IS department's equipment and personnel to help the organisation achieve its goals. Senior manager concerned with the overall needs of the organisation. Need both technical and business skills.
					\item[LAN Administrators] Set up and manage the network, hardware, software, and security processes.
						\begin{description}
							\item[Local Area Network (LAN)] A computer network that connects computer systems and devices within a small area, such as an office, home, or several floors in a building.
						\end{description}
					\item[Database Administrators] Manage the use, maintenance, and security of a company's databases.
					\item[System Developers] Design and write software.
				\end{description}
				\begin{definition}{Certification}
					A process for testing skills and knowledge which results in a statement by the certifying authority that an individual is capable of performing a particular kind of job.
				\end{definition}
		\section{Exercises}
			\begin{exercise}{Self-Assessment}
				\begin{enumerate}
					\item A minimum number of management layers results in a \concept{flat} organisation structure.
					\item Giving employees and their managers more responsibility and authority is known as \concept{empowerment}.
					\item Outsourcing computing resources to a third party is known as \concept{on-demand computing}.
					\item The radical redesign of business processes is known as \concept{reengineering}.
					\item \concept{Key-operational applications} are information systems that an organisation could not operate without.
					\item Productivity equals \concept{output divided by input}.
					\item If a firm can outperform its industry, it has a \concept{competitive advantage}.
					\item \concept{Market share} is the percentage of sales that a product or service has in relation to the total market.
					\item An agreement between two or more companies is known as a \concept{strategic alliance}.
					\item The head of the information systems department is the \concept{Chief Information Officer}.
				\end{enumerate}
			\end{exercise}
			\begin{exercise}{Review Questions}
				\begin{enumerate}
					\question{What is the value chain?}
					A series (chain) of activities that involves the elements used to add value to raw materials in a produced product or service.
					\question{Describe a virtual organisation structure}
					A structure where the members of the organisation are in different places geographically.
					\question{What is technology diffusion}
					A measure of how widely technology is spread throughout an organisation.
					\question{Describe on-demand computing}
					Outsourcing the computer resources needed for an organisation's information systems.
					\question{Define Continuous Improvement and compare it with reengineering}
					Continuous improvement means making continual minor changes in order to improve business processes and add value to products and services. Reengineering is redesigning the entire system, in order to achieve a major breakthrough.
					\question{What is a support application?}
					An unessential application that is used to make work more convenient.
					\question{Describe some ways in which IS can be evaluated.}
					Measuring Productivity, or Return on Investment (ROI).
					\question{Why would a company constantly develop new products?}
					In order to achieve or retain a competitive advantage.
				\end{enumerate}
			\end{exercise}
	\vbox{\rulechapterend}
\end{document}