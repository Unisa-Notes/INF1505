\providecommand{\main}{../..}
\documentclass[\main/notes.tex]{subfiles}

\begin{document}
	\setcounter{chapter}{12}
	\chapter{Security, Privacy and Ethical Issues in Information Systems}
	\chaptermark{Security, Privacy and Ethics}
		\section{Computer Waste and Mistakes}
			\begin{definition}{Computer Waste}
				The inappropriate use of computer technology and resources.

				Includes employees wasting computer resources and time by playing games and surfing the web, sending unnecessary email, printing documents and other materials that is then not read, developing systems that are then not used to their full extent, and discarding old hardware that could be recycled. Junk email (\concept{spam}) is wasteful.

				Waste is caused by the improper management of information systems and resources.
			\end{definition}
			\begin{definition}{Computer-Related Mistakes}
				Errors, failures, and other computer problems that make computer output incorrect, or not useful, caused mostly by human error.

				Can be caused by unclear expectations and a lack of feedback.
			\end{definition}
			\begin{sidenote}{Types of Computer-Related Mistakes}
				\begin{itemize}[nosep]
					\item Data entry or data capture errors
					\item Errors in computer programs
					\item Errors in file handling
					\item Mishandling of computer output
					\item Inadequate planning for and control of equipment malfunctions
					\item Inadequate planning for and control of environmental difficulties
					\item Installing computer capacity inadequate for the level of activity on corporate websites
					\item Failure to provide access to the most current information, by not adding new URLs, and not deleting old URLs.
				\end{itemize}
			\end{sidenote}
			\subsection{Preventing Computer-Related Waste and Mistakes}
				\begin{enumerate}[nosep]
					\item Establish policies and procedures
					\item Implement policies and procedures
					\item Monitor policies and procedures
					\item Review policies and procedures
				\end{enumerate}
				\subsubsection{Establishing Policies and Procedures}
					\begin{sidenote}{Preventing Waste}
						Establish policies and procedures regarding efficient acquisition, use, and disposal of systems and devices.

						Stringent policies on acquisition of computer systems and equipment, including a formal justification statement, the definition of standard computing platforms, and the use of preferred vendors for all acquisitions.
					\end{sidenote}
					\begin{sidenote}{Preventing Error}
						\begin{itemize}[nosep]
							\item Training programs for individuals and workgroups.
							\item Manuals and documents on how computer systems are to be maintained and used.
							\item Needing approvals for certain systems and applications before they are implemented and used, to ensure compatibility and cost-effectiveness
							\item A requirement that documentation and descriptions of certain applications be submitted to a central office.
						\end{itemize}
					\end{sidenote}
				\subsubsection{Implementing Policies and Procedures}
					Varies according to the type of business. Typically focus on the implementation of source data automation, the use of data editing to ensure accuracy and completeness, and the assignment of responsibility for data accuracy within each information system.
				\subsubsection{Monitoring Policies and Procedures}
					Monitor routine practices, and take corrective action if necessary. Can be done using internal audits to measure actual results against established goals.
				\subsubsection{Reviewing Policies and Procedures}
					Determine whether the existing policies and procedures are adequate. Should allow companies to take a proactive approach.

		\section{Computer Crime}
			\subsection{Identity Theft}
				\begin{definition}{Identity Theft}
					An imposter obtains key pieces of personal identification information, and uses them to open bank accounts, get credit cards, loans, benefits, and documents such as passports and driving licences, in the victim's name.
				\end{definition}
				\begin{definition}{Social Engineering}
					Using one's social skills to get computer users to provide you with information to access an information system or its data.
					\begin{description}
						\item[Shoulder Surfing] An identity thief stands next to someone at a public office, and watches as the person fills out personal information on a form.
					\end{description}
				\end{definition}
			\subsection{Cyberterrorism}
				\begin{definition}{Cyberterroist}
					Someone who intimidates a government or organisation, to advance his or her political or social objectives, by launching computer-based attacks against computers, networks, and the information stored on them.
				\end{definition}
			\subsection{Illegal Access and Use}
				\begin{definition}{Cracker (or Hacker)}
					A computer-savvy person who attempts to gain unauthorised or illegal access to computer systems.
				\end{definition}
				\begin{definition}{Script Kiddies}
					Crackers with little technical savvy, who download programs called \concept{scripts}, which automate the job of breaking into computers.
				\end{definition}
				\begin{definition}{Insiders}
					Employees, disgruntled or otherwise, working solo or in concert with outsiders, to compromise corporate systems.
				\end{definition}
				\subsubsection{Malware}
					\begin{definition}{Malware}
						Software programs that, when loaded into a computer system, will destroy, interrupt, or cause errors in processing.
					\end{definition}
					\begin{definition}{Virus}
						A computer program file capable of attaching to discs or other files, and replicating itself repeatedly, typically without the user's knowledge or permission.
					\end{definition}
					\begin{definition}{Worm}
						A parasitic computer program that can create copies of itself on the infected computer, or send copies to other computers via a network. Unlike a virus, they do not infect other computer program files.
					\end{definition}
					\begin{definition}{Trojan Horse}
						A malicious program that disguises itself as a useful application, and purposefully does something that the user does not expect. Not viruses, as they do not replicate.
						\begin{description}
							\item[Spyware] Software which records all manner of personal information about users, and forwards it to the spyware's owner, all without the user's consent.
							\item[Logic Bomb] A type of Trojan Horse than executes when specific conditions occur.
						\end{description}
					\end{definition}
					\begin{definition}{Variant}
						A modified version of a virus, that is produced by the virus' author, or another person who amends the original virus code.
					\end{definition}
					\begin{definition}{Antivirus Program}
						Software that runs in the background, to protect your computer from dangers lurking on the Internet and other possible sources of infected files.
					\end{definition}
				\subsubsection{Equipment Theft}
					To fight the theft of computer equipment, and data on the computer equipment, many companies use devices that disable the disc drive, and/or lock the computer to the desk.
				\subsubsection{Software and Internet Software Piracy}
					\begin{definition}{Software Piracy}
						The act of illegally duplicating software.
					\end{definition}

		\section{Preventing Computer-Related Crime}
			\subsection{Crime Prevention by the State}
				Many countries have passed laws that govern when and how data about individuals can be stored and processed.
			\subsection{Crime Prevention by Organisations}
				Many businesses have designed procedures and specialised hardware and software to protect their corporate data and systems. Specialised hardware and software, such as \concept{encryption devices}, can be used to encode data and information to help prevent unauthorised use.
				\begin{definition}{Encryption}
					The process of converting an original electronic message into a form that can be understood only by the intended recipients.
					\begin{description}
						\item[Key] A variable value that is applied using an algorithm to a string or block of unencrypted text to produce encrypted text, or to decrypt encrypted text.
					\end{description}
					Encryption methods rely on the limitations of computing power for their effectiveness.
				\end{definition}
				\begin{definition}{Public Key Infrastructure (PKI)}
					A means to enable users of an unsecured public network, such as the Internet, to securely and privately exchange data through the use of a public and private cryptographic key pair that is obtained and shared through a trusted authority.

					The most common method on the Internet for authenticating a message sender, or encrypting a message. Uses two keys: a \concept{public key} that is available to the public, and used to send that individual encrypted messages; and a \concept{private key}, a key that is known only by the message receiver.
				\end{definition}
				\begin{definition}{Biometrics}
					The measurement of one of a person's traits, whether physical or behavioural.
				\end{definition}
			\subsection{Using Intrusion Detection Software}
				\begin{definition}{Intrusion Detection System (IDS)}
					Software that monitors system and network resources, and notifies security personnel when it senses a possible intrusion.
				\end{definition}
			\subsection{Using Managed Security Service Providers}
				\begin{definition}{Managed Security Service Providers (MSSPs)}
					Monitor, manage, and maintain network security for both hardware and software.
				\end{definition}

		\section{Privacy}
			When information is computerised, and can be processed and transferred easily, augmented and collated, summarised and reported, privacy concerns grow.
			\begin{definition}{Platform for Privacy Preferences (P3P)}
				A screening technology that shields users from websites that don't provide the level of privacy protection they desire.
			\end{definition}
			\subsection{Fairness in Information Use}
				\begin{center}
					\begin{tblr}{colspec={|>{\raggedright}X| >{\raggedright}X>{\raggedright}X|}, row{1}={font=\bfseries}, column{1}={font=\bfseries}, row{even}={table even}}
						\toprule
						Fairness Issues & Database Storage & Database Usage\\
						\midrule
						The right to know & Knowledge & Notice\\
						The ability to decide & Control & Consent\\
						\bottomrule
					\end{tblr}
				\end{center}
				\begin{sidenote}{Fairness Issues}
					\begin{description}
						\item[Knowledge] Should people know what data is stored about them?
						\item[Control] Should people be able to correct errors in corporate database systems?
						\item[Notice] Should an organisation that uses personal data for a purpose other than the original purpose notify individuals in advance?
						\item[Consent] If information on people is to be used for other purposes, should these people be asked to give their consent before data on them is used?
					\end{description}
				\end{sidenote}

		\section[Relevant Laws]{Relevant Laws Governing the Use of Technology}
			\begin{definition}{General Data Protection Regulation (GDPR)}
				Governs how companies, charities and other organisations can collect, use, and share information, regardless of whether that information is stored electronically in a computer's memory, or on paper in a filing cabinet.

				European law, introduced in 2016.
				\begin{multicols}{2}
					\begin{itemize}[nosep]
						\item Lawfulness, fairness, and transparency
						\item Purpose limitation
						\item Data minimisation
						\item Accuracy
						\item Storage limitation
						\item Integrity and confidentiality
						\item Accountability
					\end{itemize}
				\end{multicols}
			\end{definition}
			In South Africa, a similar law is the \concept{Protection of Personal Information Act (POPI)}.

		\section{The Work Environment}
			\subsection{Health Concerns}
				For some people, working with computers can cause occupational stress. Computer use can affect physical health as well.
				\begin{sidenote}{Physical Health Impact}
					\begin{multicols}{2}
						\begin{itemize}[nosep]
							\item Strains
							\item Sprains
							\item Tendonitis
							\item Tennis Elbow
							\item Inability to hold objects
							\item Sharp pain in the fingers
							\item Repetitive Strain Injury (RSI)
							\item Carpal Tunnel Syndrome (CTS)
						\end{itemize}
					\end{multicols}
					\begin{definition}{Carpal Tunnel Syndrome (CTS)}
						The aggravation of the pathway for nerves that travel along the wrist (the carpal tunnel). Involves wrist pain, a feeling of tingling and numbness, and difficulty grasping and holding objects.
					\end{definition}
				\end{sidenote}
			\begin{definition}{Ergonomics}
				The science of designing machines, products and systems to maximise the safety, comfort and efficiency of the people who use them.
			\end{definition}

		\section{Ethical Issues in Information Systems}
			\begin{definition}{Code of Ethics}
				A code that states the principles and core values that are essential to a set of people and, therefore, govern their behaviour.
			\end{definition}

		\pagebreak
		\section{Exercises}
			\begin{exercise}{Self-Assessment}
				\begin{enumerate}
					\item Fooling someone into giving you their password is called \concept{social engineering}.
					\item Another name for a zombie network is \concept{botnet}.
					\item \concept{Antivirus} software runs in the background, protecting your computer from infected files.
					\item \concept{Software pirates} are people who illegally duplicate software.
					\item A \concept{shill} is a slang term for someone paid by a company to make them look good, and make you a victim of a scam.
					\item \concept{Public Key Infrastructure} allows you to send information securely through a public network.
					\item \concept{Biometrics} involves identifying you from your physical traits.
					\item An \concept{Intrusion Detection System} watches for people illegally using a company's network.
					\item You work email account is legally accessible only by you. \concept{False}.
					\item Under the \concept{Right to be forgotten}, an individual can ask for certain webpages to be removed from the results of web searches.
				\end{enumerate}
			\end{exercise}

	\vbox{\rulechapterend}
\end{document}