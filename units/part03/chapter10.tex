\providecommand{\main}{../..}
\documentclass[\main/notes.tex]{subfiles}

\begin{document}
	\setcounter{chapter}{9}
	\chapter{Pervasive Computing}
		\section{Introduction}
			\begin{definition}{Pervasive Computing}
				A term meaning the move of the computer away from the desktop, and towards something that is all around us, all the time. This is also known as \concept{ubiquitous computing}.
			\end{definition}
			\begin{definition}{Computer Supported Cooperative Work}
				A term that refers to technologies which allow groups to work together to achieve goals.
			\end{definition}

		\section{Wireless Internet Access}
			\begin{definition}{Wi-Fi Hotspot}
				An area where Wi-Fi wireless Internet access is available.
			\end{definition}

		\section{Mobile Devices}
			\subsection{Smartphones}
				These devices are cheaper and more robust than laptops, and can be connected with a range of accessories to increase their functionality.
				\begin{definition}{Global Positioning System (GPS)}
					A navigation system that enables a receiver to determine its precise location.
				\end{definition}
			\subsection{Wearable Technology}
				\begin{definition}{Wearable Technology}
					A term that refers to computers and computing technology that are worn on the body.
				\end{definition}
			\subsection{E-Money}
				\begin{definition}{E-Money}
					The transfer of funds electronically rather than by handing over physical coins and notes.
				\end{definition}
			\subsection{Tangible Media}
				\begin{definition}{Tangible Media}
					Represent information stored on a computer through the use of physical objects.
				\end{definition}
				\begin{definition}{Phicon}
					Stands for `physical icon', and is a physical representation of digital data, in the same way that an icon on a computer screen represents a file.
				\end{definition}
			\subsection{Personal Robotics}
				\begin{definition}{Personal Robotics}
					A term which refers to robotic companions that people socialise with.
				\end{definition}
			\subsection{Virtual Pets}
				\begin{definition}{Virtual Pet}
					An artificial companion. Could be screen based, or a robot.
				\end{definition}

		\pagebreak
		\section{Computer Supported Cooperative Work}
			\begin{definition}{Computer Supported Coooperative Work (CSCW)}
				Technologies that allow groups to work together to achieve goals. Individuals in the group can be \concept{co-located} (in the same place), or geographically separated. The work can happen \concept{synchronously} (at the same time), or \concept{asynchronously} (at different times).
			\end{definition}
			\subsection{Videoconferencing}
				\begin{definition}{Videoconference}
					A simultaneous communication between two or more parties where they both see and hear each other.
				\end{definition}
			\subsection{Messaging}
				\begin{definition}{Email}
					Asynchronous text-based communication.
				\end{definition}
				\begin{definition}{Instand Messaging}
					Used for synchronous communication -- two (or more) people are communicating at the same time, usually typing short sentences to build up a conversation.
				\end{definition}
				\begin{definition}{Chat Room}
					A facility that enables two or more people to engage in interactive `conversations' over the web.
				\end{definition}
			\subsection{Interactive Whiteboards}
				\begin{definition}{Interactive Whiteboard}
					A combination of a whiteboard and a desktop computer.

					This term can be used to mean slightly different technologies.
				\end{definition}
			\subsection{Wikis}
				\begin{definition}{Wiki}
					A web page that can be edited by anyone with the proper authority.
				\end{definition}
			\subsection{MMOGs}
				\begin{definition}{MMOGs}
					Massively Multiplayer Online Game.
				\end{definition}
				\begin{definition}{Virtual Worlds}
					A computer-based environment where users' avatars can interact.
				\end{definition}
			\subsection{Blogs, Podcasts, and Live Streaming}
				\begin{definition}{Blog}
					Short for `weblog', a combination of the words `web' and `log'. An online diary.

					A website that people create and use, to write about their observations, experiences, and feelings on a wide range of topics.
					\begin{description}[nosep]
						\item[Blogosphere] The community of blogs and bloggers.
						\item[Blogger] A person who creates a blog.
						\item[Blogging] The process of placing entries on a blog site.
					\end{description}

					A \concept{microblog} has the same goals as a normal blog, but the posts are limited in size. Twitter is an example, where the limit is 280 characters.
				\end{definition}
				\begin{definition}{Podcast}
					An audio broadcast over the Internet. Essentially an audio blog.
				\end{definition}
				\begin{definition}{Really Simple Syndication}
					A collection of web formats to help provide web content, or summaries of web content. Used to offer automatic updates for blogs and podcasts.
				\end{definition}
				\begin{definition}{Live Streaming}
					Allows users to record and publish videos at the same time.
				\end{definition}
			\subsection{Cloud Tools}
				Users can access software via a browser on any of their devices, making it fully mobile, and the software connects users, which allows computer supported cooperative work.

		\section[More Applications of E-Commerce and m-Commerce]{More Applications of Electronic and Mobile Commerce}
			\subsection{Retail and Wholesale}
				\begin{definition}{Electronic Retailing (e-tailing)}
					The direct sale from business to consumer through electronic shopfronts, typically designed around an electronic catalogue and shopping cart model.

					\begin{description}
						\item[Cybermall] A single website that offers many products and services at one Internet location.
					\end{description}

					A key sector of wholesale e-commerce is spending on \concept{manufacturing, repair and operations (MRO)} of goods and services.
				\end{definition}
			\subsection{Manufacturing}
				\begin{definition}{Electronic Exchange}
						An electronic forum where manufacturers, suppliers, and competitors buy and sell goods, trade market information, and run back-office operations, such as inventory control.

						The business centre is not a physical building, but rather a network-based location where business interactions occur.
						\begin{description}
							\item[Private exchange] An exchange owned and operated by a single company.
							\item[Public exchange] An exchange owned and operated by industry groups.  
						\end{description}
				\end{definition}
			\subsection{Marketing}
				\begin{definition}{Market Segmentation}
					The identification of specific markets to target them with advertising messages. Divides the pool of potential customers into subgroups.
				\end{definition}
				\begin{definition}{Technology-enabled relationship management}
					Occurs when a firm obtains detailed information about a customer's behaviour, preferences, needs, and buying patterns. A firm uses this information to set prices, negotiate terms, tailor promotions, add product features, and otherwise customise its entire relationship with that customer.
				\end{definition}
			\subsection{Investment and Finance}
				\begin{definition}{Electronic Bill Presentment}
					A method of billing whereby a vendor posts an image of your statement on the Internet, and alerts you by email that your bill has arrived.
				\end{definition}
			\subsection{Advantages of Electronic and Mobile Commerce}
				\begin{sidenote}{Advantages of Electronic and Mobile Commerce}
					\begin{itemize}[nosep]
						\item Global reach
						\item Reduce costs
						\item Speed the flow of goods and information
						\item Increase accuracy
						\item Improve customer service
					\end{itemize}
				\end{sidenote}

		\section{Exercises}
			\begin{exercise}{Self-Assessment}
				\begin{enumerate}
					\item What is a hotspot? \concept{An area where wireless Internet access is available}.
					\item `Smart shoes' would be an example of \concept{wearable technology}.
					\item Paying without cash is often labelled \concept{e-money}.
					\item M-Pesa is attempting to replace a bank account with a \concept{mobile phone}.
					\item A \concept{virtual pet} attempts to get a user to feel emotionally attached to it, and so continue to interact with it.
					\item CSCW stands for \concept{Computer Supported Cooperative Work}.
					\item A device that allows notes on a whiteboard to be saved is an \concept{interactive whiteboard}.
					\item An online diary is often called a \concept{blog}.
					\item `On the move retail' is sometimes called \concept{m-commerce}.
					\item Advertising to particular market segments is known as \concept{market segmentation}.
				\end{enumerate}
			\end{exercise}

	\vbox{\rulechapterend}
\end{document}