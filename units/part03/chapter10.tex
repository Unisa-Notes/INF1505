\providecommand{\main}{../..}
\documentclass[\main/notes.tex]{subfiles}

\begin{document}
	\setcounter{chapter}{9}
	\chapter{Pervasive Computing}
		\section{Introduction}
			\begin{definition}{Pervasive Computing}
				A term meaning the move of the computer away from the desktop, and towards something that is all around us, all the time. This is also known as \concept{ubiquitous computing}.
			\end{definition}
			\begin{definition}{Computer Supported Cooperative Work}
				A term that refers to technologies which allow groups to work together to achieve goals.
			\end{definition}

		\section{Wireless Internet Access}
			\begin{definition}{Wi-Fi Hotspot}
				An area where Wi-Fi wireless Internet access is available.
			\end{definition}

		\section{Mobile Devices}
			\subsection{Smartphones}
				These devices are cheaper and more robust than laptops, and can be connected with a range of accessories to increase their functionality.
				\begin{definition}{Global Positioning System (GPS)}
					A navigation system that enables a receiver to determine its precise location.
				\end{definition}
			\subsection{Wearable Technology}
				\begin{definition}{Wearable Technology}
					A term that refers to computers and computing technology that are worn on the body.
				\end{definition}
			\subsection{E-Money}
				\begin{definition}{E-Money}
					The transfer of funds electronically rather than by handing over physical coins and notes.
				\end{definition}
			\subsection{Tangible Media}
				\begin{definition}{Tangible Media}
					Represent information stored on a computer through the use of physical objects.
				\end{definition}
				\begin{definition}{Phicon}
					Stands for `physical icon', and is a physical representation of digital data, in the same way that an icon on a computer screen represents a file.
				\end{definition}
			\subsection{Personal Robotics}
				\begin{definition}{Personal Robotics}
					A term which refers to robotic companions that people socialise with.
				\end{definition}
			\subsection{Virtual Pets}
				\begin{definition}{Virtual Pet}
					An artificial companion. Could be screen based, or a robot.
				\end{definition}

		\pagebreak
		\section{Computer Supported Cooperative Work}
			\begin{definition}{Computer Supported Coooperative Work (CSCW)}
				Technologies that allow groups to work together to achieve goals. Individuals in the group can be \concept{co-located} (in the same place), or geographically separated. The work can happen \concept{synchronously} (at the same time), or \concept{asynchronously} (at different times).
			\end{definition}
			\subsection{Videoconferencing}
				\begin{definition}{Videoconference}
					A simultaneous communication between two or more parties where they both see and hear each other.
				\end{definition}
			\subsection{Messaging}
				\begin{definition}{Email}
					Asynchronous text-based communication.
				\end{definition}
				\begin{definition}{Instand Messaging}
					Used for synchronous communication -- two (or more) people are communicating at the same time, usually typing short sentences to build up a conversation.
				\end{definition}
				\begin{definition}{Chat Room}
					A facility that enables two or more people to engage in interactive `conversations' over the web.
				\end{definition}
			\subsection{Interactive Whiteboards}
				\begin{definition}{Interactive Whiteboard}
					A combination of a whiteboard and a desktop computer.

					This term can be used to mean slightly different technologies.
				\end{definition}
			\subsection{Wikis}
				\begin{definition}{Wiki}
					A web page that can be edited by anyone with the proper authority.
				\end{definition}
			\subsection{MMOGs}
				\begin{definition}{MMOGs}
					Massively Multiplayer Online Game.
				\end{definition}
				\begin{definition}{Virtual Worlds}
					A computer-based environment where users' avatars can interact.
				\end{definition}
			\subsection{Blogs, Podcasts, and Live Streaming}
				\begin{definition}{Blog}
					Short for `weblog', a combination of the words `web' and `log'. An online diary.

					A website that people create and use, to write about their observations, experiences, and feelings on a wide range of topics.
					\begin{description}[nosep]
						\item[Blogosphere] The community of blogs and bloggers.
						\item[Blogger] A person who creates a blog.
						\item[Blogging] The process of placing entries on a blog site.
					\end{description}

					A \concept{microblog} has the same goals as a normal blog, but the posts are limited in size. Twitter is an example, where the limit is 280 characters.
				\end{definition}
				\begin{definition}{Podcast}
					An audio broadcast over the Internet. Essentially an audio blog.
				\end{definition}
				\begin{definition}{Really Simple Syndication}
					A collection of web formats to help provide web content, or summaries of web content. Used to offer automatic updates for blogs and podcasts.
				\end{definition}
				\begin{definition}{Live Streaming}
					Allows users to record and publish videos at the same time.
				\end{definition}
			\subsection{Cloud Tools}
				Users can access software via a browser on any of their devices, making it fully mobile, and the software connects users, which allows computer supported cooperative work.

	\vbox{\rulechapterend}
\end{document}