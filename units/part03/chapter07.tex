\providecommand{\main}{../..}
\documentclass[\main/notes.tex]{subfiles}

\begin{document}
	\setcounter{chapter}{6}
	\chapter{Operational Systems}
		\section{Enterprise Resource Planning}
			\begin{definition}{Enterprise Resource Planning (ERP)}
				A system that manages an entire company's vital business information.

				Evolved from \concept{material requirements planning (MRP)}, which allowed companies to plan out how much raw material they would need at a certain time in the future, plan their production, control their inventory, and manage their purchasing process.
			\end{definition}
			\begin{sidenote}{Advantages of ERP Systems}
				\begin{description}
					\item[Improved Access to Data for Operational Decision-Making] ERP systems use an integrated database -- use one set of data to support all business functions.
					\item[Elimination of Costly, Inflexible Legacy Systems] Replace separate systems with one, single, integrated system for the entire enterprise.
					\item[Improvement of Work Processes] As ERP development is designed to support best practices, this ensures the processes are based on these.
					\item[Upgrade of Technology Infrastructure] An organisation can eliminate the multiple hardware platforms, operating systems, and databases it is currently using. This reduces ongoing maintenance.
				\end{description}
			\end{sidenote}
			\pagebreak
			\begin{sidenote}{Disadvantages of ERP Systems}
				\begin{description}
					\item[Expense and Time in Implementation] Requires time and money in order to correctly set it up, and implement it
					\item[Difficulty Implementing Change] An organisation might require radical change in the way it operates in order to use the ERP System
					\item[Difficulty Integrating With Other Systems] Hard to integrate existing systems with a new ERP system
					\item[Difficulty in Loading Data into New ERP System] Need to load data from existing systems into the new one. This is dependent on the scope of ERP implementation.
						\begin{description}
							\item[Data mapping] The examination of each data item required for the new ERP system, and determining where the data item will come from. Often requires \concept{data clean-up} afterwards.
							\item[Data loading] Performed either by using data conversion software, or by end-users entering data via input screens of the new system.
						\end{description}
					\item[Risks in Using One Vendor] After an organisation has switched to a vendor for an ERP system, the vendor has less incentive to listen and respond to customer concerns.
					\item[Risk of Implementation Failure] Implementation requires large amounts of resources, the best IS and business people, and plenty of management support.
				\end{description}
			\end{sidenote}
			\subsection{ERP for Small- and Medium-Sized Enterprises (SMEs)}
				SMEs (both for-profit and not-for-profit) can achieve benefits using ERP. Many of these elect to use \concept{open-source ERP systems}, where the source code can be seen and modified.
				\begin{center}
					\begin{tblr}{colspec={cl}, row{1}={font=\bfseries}}
						\toprule
						Vendor & ERP Solutions\\
						\midrule
						Apache & Open For Business ERP\\
						Compiere & Compiere Open Source ERP\\
						Openbravo & Openbravo Open Source ERP\\
						WebERP & WebERP\\
						\bottomrule
					\end{tblr}
				\end{center}
	\vbox{\rulechapterend}
\end{document}