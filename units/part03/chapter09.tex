\providecommand{\main}{../..}
\documentclass[\main/notes.tex]{subfiles}

\begin{document}
	\setcounter{chapter}{8}
	\chapter{Knowledge Management and Specialised Information Systems}
	\chaptermark{Knowledge Management Systems}
		\section{Knowledge Management Systems}
			\begin{definition}{Knowledge}
				The awareness and understanding of a set of information, and the ways that information can be made useful, to support a specific task or reach a decision.
			\end{definition}
			\begin{definition}{Knowledge Management System (KMS)}
				An organised collection of people, procedures, software, databases and devices, used to create, store, share and use the organisation's knowledge and experience.
				\begin{indentparagraph}
					\begin{description}
						\item[Explicit Knowledge] Knowledge that is objective, and can be measured in reports, papers, and rules.
						\item[Tacit Knowledge] Knowledge that is hard to measure and document, and typically is not objective or formalised.
					\end{description}
					Many organisations actively attempt to convert tacit knowledge to explicit knowledge, to make the knowledge easier to measure, document, and share with others.
				\end{indentparagraph}
			\end{definition}
			\begin{sidenote}{Creating Knowledge}
				\begin{enumerate}
					\item When an individual learns directly from another individual, in an apprentice type relationship, tacit knowledge is created from tacit knowledge.
					\item When two pieces of explicit knowledge are combined.
					\item When an expert writes a book teaching others, explicit knowledge is being created from tacit knowledge.
					\item When someone reads that book, and (eventually) becomes an expert themselves, tacit knowledge has been created by explicit knowledge.
				\end{enumerate}
			\end{sidenote}
			\subsection{Obtaining, Storing, Sharing, and Using Knowledge}
				\begin{definition}{Chief Knowledge Officer}
					A top-level executive who helps the organisation use a KMS to create, store and use knowledge, to achieve organisational goals.
				\end{definition}
				After knowledge is created, it is often stored in a \concept{knowledge repository}.

				Using a KMS begins with locating the organisation's knowledge. This is often done using a \concept{knowledge map or directory} that points the knowledge worker to the needed knowledge.
	\vbox{\rulechapterend}
\end{document}