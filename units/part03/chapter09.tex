\providecommand{\main}{../..}
\documentclass[\main/notes.tex]{subfiles}

\begin{document}
	\setcounter{chapter}{8}
	\chapter{Knowledge Management and Specialised Information Systems}
	\chaptermark{Knowledge Management Systems}
		\section{Knowledge Management Systems}
			\begin{definition}{Knowledge}
				The awareness and understanding of a set of information, and the ways that information can be made useful, to support a specific task or reach a decision.
			\end{definition}
			\begin{definition}{Knowledge Management System (KMS)}
				An organised collection of people, procedures, software, databases and devices, used to create, store, share and use the organisation's knowledge and experience.
				\begin{indentparagraph}
					\begin{description}
						\item[Explicit Knowledge] Knowledge that is objective, and can be measured in reports, papers, and rules.
						\item[Tacit Knowledge] Knowledge that is hard to measure and document, and typically is not objective or formalised.
					\end{description}
					Many organisations actively attempt to convert tacit knowledge to explicit knowledge, to make the knowledge easier to measure, document, and share with others.
				\end{indentparagraph}
			\end{definition}
			\begin{sidenote}{Creating Knowledge}
				\begin{enumerate}
					\item When an individual learns directly from another individual, in an apprentice type relationship, tacit knowledge is created from tacit knowledge.
					\item When two pieces of explicit knowledge are combined.
					\item When an expert writes a book teaching others, explicit knowledge is being created from tacit knowledge.
					\item When someone reads that book, and (eventually) becomes an expert themselves, tacit knowledge has been created by explicit knowledge.
				\end{enumerate}
			\end{sidenote}
			\subsection{Obtaining, Storing, Sharing, and Using Knowledge}
				\begin{definition}{Chief Knowledge Officer}
					A top-level executive who helps the organisation use a KMS to create, store and use knowledge, to achieve organisational goals.
				\end{definition}
				After knowledge is created, it is often stored in a \concept{knowledge repository}.

				Using a KMS begins with locating the organisation's knowledge. This is often done using a \concept{knowledge map or directory} that points the knowledge worker to the needed knowledge.

		\section{Artificial Intelligence}
			\begin{definition}{Artificial Intelligence}
				Computers with the ability to mimic or duplicate the functions of the human brain.
			\end{definition}
			\begin{definition}{Artificial Intelligence Systems}
				People, procedures, hardware, software, data, and knowledge needed to develop computer systems and machines that demonstrate characteristics of intelligence.
			\end{definition}
			\subsection{The Nature of Intelligence}
				\begin{definition}{Intelligent Behaviour}
					The ability to:
					\begin{itemize}
						\item Learn from experience and apply the knowledge acquired from experience.
						\item Handle complex situations.
						\item Solve problems when important information is missing.
						\item Determine what is important.
						\item React quickly and correctly to a new situation.
						\item Understand visual images.
							\begin{indentparagraph}
								\begin{description}
									\item[Perceptive System] A system that approximates the way a person sees, hears, and feels objects.
								\end{description}
							\end{indentparagraph}
						\item Process and manipulate symbols
						\item Be creative and imaginative
						\item Use heuristics
					\end{itemize}
				\end{definition}
			\pagebreak
			\subsection{The Difference Between Natural and Artificial Intelligence}
				\begin{center}
					\begin{tblr}{colspec={ | >{\raggedright}X[3]|X[1]|X[1]|X[1]|X[1]| }, row{1}={font=\bfseries}, row{2}={font=\bfseries}, row{odd[2]}={table even}}
						\toprule
						\SetCell[r=2]{c} Ability to
							  & \SetCell[c=2]{c} Natural Intelligence
								    &     & \SetCell[c=2]{c} Artificial Intelligence & \\
							  & \SetCell[c=1]{c}Low & \SetCell[c=1]{c}High & \SetCell[c=1]{c}Low & \SetCell[c=1]{c}High\\
						\midrule
						Use sensors &  & \SetCell[c=1]{c}X & \SetCell[c=1]{c}X & \\
						Be creative and imaginative &  & \SetCell[c=1]{c}X & \SetCell[c=1]{c}X & \\
						Learn from experience &  & \SetCell[c=1]{c}X & \SetCell[c=1]{c}X & \\
						Adapt to new situations &  & \SetCell[c=1]{c}X & \SetCell[c=1]{c}X & \\
						Afford the cost of acquiring intelligence &  & \SetCell[c=1]{c}X & \SetCell[c=1]{c}X & \\
						Acquire a large amount of external information & & \SetCell[c=1]{c}X & & \SetCell[c=1]{c}X \\
						Use a variety of information sources & & \SetCell[c=1]{c}X & & \SetCell[c=1]{c}X \\
						Make complex calculations & \SetCell[c=1]{c}X & & & \SetCell[c=1]{c}X\\
						Transfer information & \SetCell[c=1]{c}X & & & \SetCell[c=1]{c}X\\
						Make a series of calculations rapidly and accurately & \SetCell[c=1]{c}X & & & \SetCell[c=1]{c}X \\
						\bottomrule
					\end{tblr}
				\end{center}
			\subsection{The Major Branches of Artificial Intelligence}
				\begin{definition}{Expert Systems}
						Hardware and software that stores knowledge, and makes inferences, similar to a human expert.
				\end{definition}
				\begin{definition}{Robotics}
					Mechanical or computer devices that perform tasks requiring a high degree of precision, or that are tedious or hazardous for humans.

					Robots are used to the `three Ds' -- dull, dirty, and dangerous jobs.
				\end{definition}
				\begin{definition}{Vision Systems}
					The hardware and software that permit computers to capture, store, and manipulate visual images.
				\end{definition}
				\begin{definition}{Natural Language Processing}
					Processing that allows the computer to understand and react to statements and commands made in a `natural' language, such as English.
					\begin{indentparagraph}
						\begin{description}
							\item[Voice recognition] Converting sound waves into words.
						\end{description}
					\end{indentparagraph}
				\end{definition}
				\begin{definition}{Learning Systems}
					A combination of software and hardware that allows the computer to change how it functions, or react to situations based on feedback it receives.
				\end{definition}
				\begin{definition}{Neural Networks}
					A computer system that attempts to simulate the function of a human brain.

					Uses massive parallel processors in an architecture that is based on the human brain's own mesh-like structure.
				\end{definition}
				\begin{definition}{Genetic Algorithm}
					Also called a \concept{genetic program}. An approach to solving large, complex problems, in which a number of related operations or models change and evolve until the best one emerges.
				\end{definition}
				\begin{definition}{Intelligent Agent}
					Also called an \concept{intelligent robot}, or \concept{bot}. Programs and a knowledge base used to perform a specific task for a person, a process, or another program.
				\end{definition}

	\vbox{\rulechapterend}
\end{document}